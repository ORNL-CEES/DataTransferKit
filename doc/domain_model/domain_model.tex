\documentclass[letterpaper]{article}
\usepackage[top=1.0in,bottom=1.0in,left=1.0in,right=1.0in]{geometry}
\usepackage{verbatim}
\usepackage{amssymb}
\usepackage{graphicx}
\usepackage{longtable}
\usepackage{amsfonts}
\usepackage{amsmath}
\usepackage[usenames]{color}
\usepackage[
naturalnames = true, 
colorlinks = true, 
linkcolor = black,
anchorcolor = black,
citecolor = black,
menucolor = black,
urlcolor = blue
]{hyperref}
\usepackage{listings}
\usepackage{textcomp}
\definecolor{listinggray}{gray}{0.9}
\definecolor{lbcolor}{rgb}{0.9,0.9,0.9}

%%---------------------------------------------------------------------------%%
\author{Stuart R. Slattery
\\ \href{mailto:sslattery@wisc.edu}{\texttt{sslattery@wisc.edu}}
}

\date{June 22, 2012}
\title{Data Transfer Kit Domain Model}
\begin{document}
\maketitle

%%---------------------------------------------------------------------------%%
\section{Introduction}
In many physics applications, it is often desired to transfer fields
(i.e. degrees of freedom or other data) between meshes that may or may
not correlate in physical space. In addition, for massively parallel
simulations, it is typically that meshes not only do not correlate
spatially, but also that their parallel decompositions do not
correlate and are indepndent of one another. As an example, this
situation can occur in multiphysics simulations where physics fields
provide feedback between solutions or adaptive mesh simulations where
fields must be moved between meshes after refining and coarsening.

%%---------------------------------------------------------------------------%%
\section{Mesh}

\subsection{Topology}

\subsection{Geometric Rendezvous}

%%---------------------------------------------------------------------------%%
\section{Fields}

\subsection{Field Evaluations}

%%---------------------------------------------------------------------------%%
\section{Mesh/Field Mapping}

%%---------------------------------------------------------------------------%%
\section{Transfer Operations}

%%---------------------------------------------------------------------------%%
\section{Conclusion}

%%---------------------------------------------------------------------------%%
\pagebreak
\bibliographystyle{ieeetr}
\bibliography{references}
\end{document}


