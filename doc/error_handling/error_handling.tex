\documentclass[letterpaper]{article}
\usepackage[top=1.0in,bottom=1.0in,left=1.0in,right=1.0in]{geometry}
\usepackage{verbatim}
\usepackage{amssymb}
\usepackage{graphicx}
\usepackage{longtable}
\usepackage{amsfonts}
\usepackage{amsmath}
\usepackage[usenames]{color}
\usepackage[
naturalnames = true, 
colorlinks = true, 
linkcolor = black,
anchorcolor = black,
citecolor = black,
menucolor = black,
urlcolor = blue
]{hyperref}
\usepackage{listings}
\usepackage{textcomp}
\definecolor{listinggray}{gray}{0.9}
\definecolor{lbcolor}{rgb}{0.9,0.9,0.9}
\lstset{
  backgroundcolor=\color{lbcolor},
  tabsize=4,
  rulecolor=,
  language=c++,
  basicstyle=\scriptsize,
  upquote=true,
  aboveskip={1.5\baselineskip},
  columns=fixed,
  showstringspaces=false,
  extendedchars=true,
  breaklines=true,
  prebreak =
  \raisebox{0ex}[0ex][0ex]{\ensuremath{\hookleftarrow}},
  frame=single,
  showtabs=false,
  showspaces=false,
  showstringspaces=false,
  identifierstyle=\ttfamily,
  keywordstyle=\color[rgb]{0,0,1},
  commentstyle=\color[rgb]{0.133,0.545,0.133},
  stringstyle=\color[rgb]{0.627,0.126,0.941},
}

\author{Stuart R. Slattery
\\ \href{mailto:sslattery@wisc.edu}{\texttt{sslattery@wisc.edu}}
}

\date{March 24, 2012}
\title{Data Transfer Kit Error Handling Policy}
\begin{document}
\maketitle
\section{Introduction}
The level 3 milestone deliverable report for VRI.PSS.P4.02 includes in
Appendix C a review of the data transfer interfaces and the associated
software in the Data Transfer Kit. This review is summarized with 10
items that help define a development path forward for the Data
Transfer Kit. Item 9 of this list states the need for a clearly
defined error handling scheme that defines how exceptions are thrown
and handled at the user level. To address this issue, this document
defines an error handling scheme strongly based on the error handling
and exceptions guidlines in \cite{Sutter_2004}.

\section{Exception Scheme}
The Data Transfer Kit (DTK) will rely on C++ exceptions as the
primary means of propagating error information to the user. Per the
discussion in \cite{Sutter_2004}, all runtime exceptions thrown by DTK
will inherit from std::runtime\_error, allowing users to isolate
errors of this type with try/catch blocks. Internally, these runtime
errors are divided into three categories; precondition exceptions that
arise from a function's inability to meet preconditions, postcondition
exceptions resulting from a function's failure to achieve a
postcondition, and invariant exceptions resulting from a function's
inability to establish an ivariant. Throughout DTK, a user should
expect these exceptions to be used to test these three fundamental
conditions and for runtime exceptions to be thrown if they are
violated. In addition, when an exception is thrown, a user should
expect DTK to provide a message detailing why the exception was
thrown.

For developers implementing the data transfer interfaces defined
within DTK, it is expected that these exceptions are used to verify
the three runtime conditions within their implementation. Doing so
will then present a consistent exception scheme at the boundaries of
DTK  for users implementing the DTK API that are neither DTK
developers or DTK interface developers. In addition, throwing these
exceptions in the interface implementations will facilitate error
propagation from the interface implementations to the user through the
various API calls. 

Internally, DTK relies on the Trilinos packages Teuchos and Tpetra,
both of which throw C++ exceptions through their own schemes. Calls to
these libraries within DTK will not be isolated with try/catch
blocks. Instead, their exceptions will be propagated through the API
boundary.

\pagebreak
\bibliographystyle{ieeetr}
\bibliography{references}
\end{document}


